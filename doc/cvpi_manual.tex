% [calc-mode: plain: t]
% [calc-mode: language: latex]
\documentclass[12pt]{report}

\def\byX {\!\!\times\!\!}
\usepackage{mdframed}
\usepackage{listings}
\usepackage[letterpaper, margin=1in]{geometry}
\usepackage{graphicx}
\usepackage[breaklinks]{hyperref}
\usepackage{url}
\usepackage{breakurl}
\usepackage{amsmath}
\usepackage{amsthm}
\usepackage{hyphenat}
\usepackage{float}
\usepackage{subfig}
\author{Devin Homan}
\date{Version 0.1}
\title{CVPI\\A Computer Vision Library\\For Mobile and Embedded Platforms}
\begin{document}
\lstset{language=C}

\maketitle
\tableofcontents
\null\vfill
\noindent
Copyright \copyright  2015. See the AUTHORS document for attribution.\\
    Permission is granted to copy, distribute and/or modify this document
    under the terms of the GNU Free Documentation License, Version 1.3
    or any later version published by the Free Software Foundation;
    with no Invariant Sections, no Front-Cover Texts, and no Back-Cover Texts.
    A copy of the license is included in the section entitled "GNU
    Free Documentation License".
\newpage
\chapter{Introduction}
\label{sec-1}
CVPI is a library for implementing computer vision programs on
computers supporting OpenVG. It adds additional image processing
capabilities to OpenVG that are necessary for computer vision, as well
a as providing an interface to setup the rendering environment.

OpenVG is a hardware accelerated C API for vector and raster 2D
graphics \cite{openvg}. It is widely supported on mobile and on
embedded platforms.
\section{Motivation}
\label{sec-1-1}
Single board computers, such as the Raspberry Pi, have become a
popular platform for robotics. These boards come with GPUs, which are
currently not being utilized in this area. The GPUs on these computers
support OpenVG and OpenGL~ES. Current computer vision software, such
as OpenCV, rely on OpenCL or CUDA, which are GPU languages that are
not supported on current mobile and embedded GPUs. Currently, to do
computer vision on mobile and embedded platforms, all computation has
to be performed by the on board CPU, or the data has to be transferred
to a remote computer for processing.  OpenVG offers the necessary
functionality to create computer vision software and is available
natively on these platforms.

The Raspberry Pi is already being used for computer vision controlled
robots and is marketed as a computing platform for teaching
programming and computer hardware concepts to secondary students. In
one such example, a high-school student built a line tracking robot
using OpenCV. The Raspberry~Pi, which normally runs at 800MHz, was
overclocked to 1GHz, and the tracking rate was between 12 to 14fps
\cite{robocup}.  Running OpenVG and other GPU code is non-trivial and
under-documented on the Raspberry~Pi. The CVPI EGL interface aims to
alleviate the complexity as much as possible. CVPI can also allow for
the same results, that the student achieved with OpenCV, with greater
efficiency by moving computation to the GPU.

CVPI could be utilized beyond the Raspberry~Pi. Many GPUs used in
smart phones and other mobile devices support OpvenVG; though Android
apparently does not have an OpenVG API implementation. There are other
competing single boards, such as pcDuino, Odroid, and Banana~Pi that
could also utilize CVPI.
\section{Existing Software Architecture}
\label{sec-1-2}
OpenVG interfaces with the kernel using EGL (Native Platform Graphics
Interface). EGL, OpenVG, and OpenGL~ES are C~APIs published by
Khronos. EGL is used to set up the rendering context: surface
dimensions, what libraries to support (OpenVG or OpenGL), pixel
format, where rendering will be done (on screen or in memory)
\cite{egl}. The Raspberry~Pi's Khronos~APIs were implemented by the
GPU chip manufacturer, Broadcom, and are BSD licensed \cite{rPiVid}.

OpenGL ES was not used in this project. OpenVG and OpenGL~ES cannot
communicate directly with each-other; they cannot share contexts
\cite[p.~7]{egl}. They could potentially communicate by reading and
writing to and from main memory. OpenVG has so far been sufficient for
implementing computer vision related algorithms.
\section{CVPI Interface and Library}
\label{sec-1-3}
CVPI has two main parts, an EGL setup and take-down interface, and a
library of image processing functions relevant to computer vision.

CVPI offers a template interface for setting up and taking down
EGL. Setting up EGL correctly is non-trivial, especially on the
Raspberry~Pi, with steps that have to be done in a certain order and
settings that are very particular. The majority of the steps are
always the same; however, there are some steps that cannot be
accounted for, which are implementation dependent, that the user must
provide.

CVPI is meant to add to OpenVG rather than exist on top of it. OpenVG
is a C API, and so CVPI was written in C, specifically C99. C also
has the benefit of portability between client languages. CVPI can be
used by programmers wishing to write their projects in other
languages. Most languages have built-in or standard methods for binding
to C functions and values.
\chapter{CVPI EGL Interface}
\label{sec-2}
% by passing function pointers to the interface

Setting up and taking down EGL is non-trivial. The file {\tt cvp\_egl\_config.h}
contains the functions
{\tt cvpi\_egl\_settings\_create}, {\tt cvpi\_egl\_settings\_check},
{\tt cvpi\_egl\_instance\_setup}, and {\tt cvpi\_egl\_instance\_takedown} to help
the programmer in this task.

{\tt cvpi\_egl\_settings\_create} creates a structure of settings
information that can be passed to {\tt cvpi\_egl\_settings\_create} or
to {\tt cvpi\_egl\_settings\_check}. {\tt cvpi\_egl\_settings\_check}
can be used to check for faulty settings in the structure. If a bad
setting is found, CVPI\_FALSE is returned and a warning is printed to
the standard error output (stderr). Every setting has associated {\tt set} and
{\tt check} functions.

{\tt cvpi\_egl\_instance\_setup} creates a cvpi\_egl\_instance structure that
can be passed to {\tt cvpi\_egl\_instance\_takedown}. The structure gives the
take-down procedure the information needed to undo the setup
procedure. The {\tt cvpi\_egl\_instance\_setup} template performs the
following steps to setup EGL on the Raspberry~Pi:
\begin{figure}[H]
\begin{mdframed}[style=default]
\begin{enumerate}
\item Calls bcm\_host\_init()
\begin{itemize}
\item Specific to the Raspberry~Pi, it is an undocumented procedure
  required for Broadcom's implementation
\end{itemize}
\item Calls eglGetDisplay()
\item Calls eglInitialize()
\begin{itemize}
\item Initializes the display returned by eglGetDisplay
\end{itemize}
\item Calls eglBindAPI()
\begin{itemize}
\item Chooses the rendering API: OpenVG or OpenGL ES
\end{itemize}
\end{enumerate}
\end{mdframed}
\caption{Algorithm for setting up EGL. \emph{(cont.)}}
\end{figure}
\pagebreak
\begin{figure}[H]
\ContinuedFloat
\begin{mdframed}[style=default]
\begin{enumerate}
\setcounter{enumi}{4}
\item Calls eglChooseConfig() twice
\begin{itemize}
\item First, to get the number of configurations supporting the client
  provided settings.  Space is allocated, and {\tt eglChooseConfig()}
  is called again to populate the space with configuration information.
\end{itemize}
\item An EGL surface is created. If the surface is a pixmap or window,
then the user must supply a function to create an EGLNativePixmapType
or an EGLNativeWindowType, respectively. These are implementation
specific types.
\begin{itemize}
\item If the user specified a pixmap surface
\begin{enumerate}
\item EGLNativePixmapType is created using a user supplied function.
\item {\tt eglCreatePixmapSurface()} uses the EGLNativePixmapType and
the\\ {\tt eglChooseConfig()} generated configuration list to create an
EGL surface. The function loops through configurations
returned by {\tt eglChooseConfig()} until one works.
\end{enumerate}
\item If the user specified a window surface
\begin{enumerate}
\item EGLNativeWindowType is created using a user supplied
  function. There are multiple Broadcom implementations for creating
  window surfaces, such as the set of functions {\tt vc\_dispmanx\_*},
  which are Broadcom specific, and OpenWF, a Khronos API for creating
  window surfaces \cite{rPiVid}.
\item {\tt eglCreateWindowSurface()} uses the EGLNativeWindowType and the\\
{\tt eglChooseConfig()} generated configuration list to create an
EGL surface. The function loops through configurations
returned by {\tt eglChooseConfig()} until one works.
\end{enumerate}
\item If the user specified a pbuffer surface
\begin{enumerate}
\item {\tt eglCreatePbufferSurface()} is called with the {\tt eglChooseConfig()}
generated configuration list to create an EGL surface. The
function loops through configurations returned by
{\tt eglChooseConfig()} until one works.
\end{enumerate}
\item If no surface type was specified, then this step is skipped.
\end{itemize}
\item Calls {\tt eglCreateContext()} to create a rendering context.
\item Calls {\tt eglMakeCurrent()} if the user specified for this context
to be the current context.
\begin{itemize}
\item The current-context {\tt read} and {\tt draw} parameters are set
  to the same surface. OpenGL~ES allows for different surfaces for
  reading and drawing; however, OpenVG does not. The user can change
  this after {\tt cvpi\_egl\_instance\_setup} finishes.
\end{itemize}
\end{enumerate}
\end{mdframed}
\caption{Algorithm for setting up EGL.}
\end{figure}

Every step in the process is logged to {\tt cvpi\_log\_file}, which
defaults to {\tt stderr}.  Finding a working EGL setup procedure for a
particular project will likely require multiple revisions. Broadcom's
EGL implementation is poorly documented and does not give explanations
why failures occur when EGL fails to start, which is in part the
reason for {\tt cvp\_egl\_config.h}.

If an EGL error occurs in {\tt cvpi\_egl\_instance\_setup}, then the
function will jump to the appropriate place in the take-down
procedure. If a system error occurs, then the GPU might be left in a
bad state that makes it impossible to create another EGL instance
without reboot. The operating system does not appear to entirely clean
up GPU related material after the program halts. The take-down
procedure is as follows:
\begin{figure}[H]
\begin{mdframed}[style=default]
\begin{enumerate}
\item Call {\tt eglDestroyContext()}.
\item Call {\tt eglDestroySurface()}.
\item If a window or pixmap was created, call a user supplied
function to undo what was done by the user supplied
EGLNativeWindowType or EGLNativePixmapType returning function.
\item Free memory allocated for {\tt eglChooseConfig()}.
\item Call {\tt eglTerminate()}.
\item Free memory allocated for {\tt cvpi\_egl\_instance\_setup}'s
  return value.
\item Call {\tt bcm\_host\_deinit()}, if using the Broadcom implementation.
\end{enumerate}
\end{mdframed}
\caption{Algorithm for taking down up EGL.}
\end{figure}
\chapter{CVPI Image Library}
\label{sec-3}
\section{OpenVG API}
\label{sec-3-1}
OpenVG has functions for manipulating images. OpenVG can read and
write using a variety of different formats. Internally, images are
represented using four 8-bit channels per pixel \cite{openvg}.

The choice of what functions to implement came from a list of
functions which are described on the web site, \underline{Image
  Processing Learning Resources} in the ``Worksheets'' subdirectory
\cite{HIPR2}. Other functions, such as histogram equalization, come
from \underline{Computer Vision} by Richard Szeliski
\cite{Szeliski}. For the most part, functions were only implemented if
they could initialize the GPU and could not be trivially and directly
done with the existing OpenVG functions. Other functions that cannot
utilize the GPU might be added to CVPI in the future. The GPU is very
proficient at performing mapping operations on data sets but is not
very good at doing reduction operations.

\subsection{CPU/GPU Memory Transfer}
\label{sec-3-1-1}
OpenVG can read data directly from CPU memory into an image using {\tt
  vgImageSubData}, and can write directly to CPU memory with {\tt
  vgGetImageSubData}. OpenVG uses RGBA (red, green, blue, alpha) to
represent the four channels \cite{openvg}. On little-endian systems,
CVPI uses VG\_sARGB\_8888 for the color space. When processing data,
returned by {\tt vgGetImageSubData}, with array indexing, the blue
channel is the first index, green is the second, red is the third, and
alpha is the last. CVPI has not yet been tested on a big-endian
system. ARM CPUs support both endiannesses but, so far, there are no
available big-endian operating system images for the Raspberry~Pi.

CVPI assumes that in memory, the image's origin is in the upper left
with data proceeding left to right and then top to bottom, which is
the standard for Video4Linux (V4L) formats
\cite[Sec.~2.1]{videoForLinux}. OpenVG's image coordinate system
starts in the lower left corner with the x-position in the horizontal
and the y-position in the vertical \cite[Sec.~10.1]{openvg}.
Essentially the y-axis is flipped and translated.  CVPI functions are
not affected by this difference, unless otherwise noted.

\subsection{Copying Images}
\label{sec-3-1-2} {\tt vgCopyImage} can copy part of an image into
another image \cite{openvg}. This function is utilized extensively by
CVPI.

\subsection{Pixel Vector Matrix Multiplication}
\label{sec-3-1-3} {\tt vgColorMatrix} multiplies a $4\byX 4$-matrix by each
pixel vector and adds a bias vector to the resulting value to
create a new image \cite{openvg}.
{\tt vgColorMatrix}'s matrix argument is the transpose of the
mathematical representation, plus four bias values.

\begin{figure}[H]
\begin{mdframed}[style=default]
\begin{lstlisting}
A[20] = {1,  2, 3, 4,
	 5,  6, 7, 8,
	 9, 10,11,12,
	 13,14,15,16,

         17,18,19,20};
\end{lstlisting}

    \[
%%% [[R'], [G'], [B'], [A']] := [[1, 5, 9, 13], [2, 6, 10, 14], [3, 7, 11, 15], [4, 8, 12, 16]] * [[R], [G], [B], [A]] + [[17], [18], [19], [20]] => [[(R + 5 * G + 9 * B + 13 * A + 17)], [(2 * R + 6 * G + 10 * B + 14 * A + 18)], [(3 * R + 7 * G + 11 * B + 15 * A + 19)], [(4 * R + 8 * G + 12 * B + 16 * A + 20)]] %%%
 \begin{pmatrix} R' \\ G' \\ B' \\ A' \end{pmatrix}
          \gets \begin{pmatrix} 1 & 5 & 9 &
 13 \\ 2 & 6 & 10 & 14 \\ 3 & 7 & 11 & 15 \\ 4 & 8 & 12 & 16 \end{pmatrix}
  \begin{pmatrix} R \\ G \\ B \\ A \end{pmatrix}
  + \begin{pmatrix} 17 \\ 18 \\ 19 \\ 20 \end{pmatrix} \to \begin{pmatrix} R + 5 G + 9 B + 13 A
  + 17 \\ 2 R + 6 G + 10 B + 14 A
            + 18 \\ 3 R + 7 G + 11 B + 15 A + 19 \\ 4 R + 8 G + 12 B + 16 A + 20 \end{pmatrix}
\]
\end{mdframed}
  \caption{{\tt vgColorMatrix} argument and mathematical representation \cite[p.~176]{openvg}.}
\end{figure}

The square matrix is an array of floats between zero and one. The
last four parameters of the input array are also floats between
zero and one. OpenVG represents pixel values as decimal numbers
between 0 and 1 when multiplying and adding, but the output is
then re-scaled to range from 0 to 255.

\subsubsection{Pre-Defined Matrices}
CVPI has a number of pre-defined inputs.
\begin{description}
\item [{cvpi\_invert\_colors}]
      \[
%%% [[-1, 0, 0, 0], [0, -1, 0, 0], [0, 0, -1, 0], [0, 0, 0, 1]] * [[R], [G], [B], [A]] + [[255], [255], [255], [0]] => [[(255 - R)], [(255 - G)], [(255 - B)], [A]] %%%
 \begin{pmatrix} -1 & 0 & 0 &
 0 \\ 0 & -1 & 0 & 0 \\ 0 & 0 & -1 & 0 \\ 0 & 0 & 0 & 1 \end{pmatrix}
          \begin{pmatrix} R \\ G \\ B \\ A \end{pmatrix}
          + \begin{pmatrix} 255 \\ 255 \\ 255 \\ 0 \end{pmatrix} \to \begin{pmatrix} 255
  - R \\ 255 - G \\ 255 - B \\ A \end{pmatrix}
\]
\end{description}

\begin{description}
\item[{cvpi\_avuy2ayuv}] Change the channel ordering returned by the function
                        {\tt cvpi\_yuyv2yuva} to canonical form.
\end{description}
    \[
%%% [[0, 0, 1, 0], [0, 1, 0, 0], [1, 0, 0, 0], [0, 0, 0, 1]] * [[V], [U], [Y], [A]] => [[Y], [U], [V], [A]] %%%
 \begin{pmatrix} 0 & 0 & 1 &
 0 \\ 0 & 1 & 0 & 0 \\ 1 & 0 & 0 & 0 \\ 0 & 0 & 0 & 1 \end{pmatrix}
          \begin{pmatrix} V \\ U \\ Y \\ A \end{pmatrix} \to \begin{pmatrix} Y \\ U \\ V \\ A \end{pmatrix}
\]

\begin{description}
\item[{cvpi\_pixel\_average}] Average all channel values together and
output the average in each channel.
\item[{cvpi\_pixel\_color\_average}] Average the color channel values together and
output the average in each color channel.
\item [{cvpi\_channel\_red}, {cvpi\_channel\_green}, {cvpi\_channel\_blue},
  {cvpi\_channel\_alpha}] Replace the values in the other
channels with the value in the specified channel.
\end{description}

\subsection{Convolution}
\label{sec-3-1-4}
OpenVG has two functions for convolution {\tt vgConvolve} and {\tt
  vgSeparableConvolve}. OpenCV also has a Gaussian blur function,
{\tt vgGaussianBlur} \cite{openvg}.

OpvenVG's convolution formula is:

\[I(x,y)=s\left( \sum_{i=0}^{w-1}\sum_{j=0}^{h-1}k(w-i-1,h-j-1)P(x+i-s_x,y+j-s_y)\right) +b\]

where $w$ and $h$ are the image's width and height, $k$ is the
convolution kernel, $P$ is the input image, $s_x$ and $s_y$ are
integer values for shifting the input and output, $s$ is a scalar, and
$b$ is a bias. Image channel values are treated
as floating point values between 0 and 1, so $b$ must be a value
between -1 and 1. The image and kernel are indexed from zero starting in the
lower left corner \cite[pp.~177-180]{openvg}.

When coding in a convolution kernel in C, the
mathematical definition of the convolution kernel must be
transposed then flipped. The kernel dimensions will also be
switched from the mathematical representation.

For example,
   \[\begin{pmatrix} 1 & 4 & 7 \\ 2 & 5 & 8 \\ 3 & 6 & 9 \end{pmatrix}^T
          \begin{pmatrix} 0 & 0 &
 1 \\ 0 & 1 &
      0 \\ 1 & 0 & 0 \end{pmatrix} \to \begin{pmatrix} 3 & 2 & 1 \\ 6 & 5 & 4 \\ 9 & 8 & 7 \end{pmatrix}
\]

\begin{lstlisting}
VGshort kernel[9] = {3,2,1,
                     6,5,4,
                     9,8,7};
\end{lstlisting}
\begin{proof}
  Given a two-dimensional C array representation of an OpenVG kernel,
  C defines the one-dimensional index of an element in a
  two-dimensional array to be $y\cdot \text{arrayHeight} + x$. The
  OpenVG kernel entry $(i,j)$ is located at $i\cdot\text{kernelHeight}
  + j$ \cite[p.~178]{openvg}.  So because $\text{arrayHeight} =
  \text{kernelHeight}$ and $\text{arrayWidth} = \text{kernelWidth}$,
  $\text{array}[x][y] = \text{kernel}(i,j)$, where $x\equiv i$ and
  $y\equiv j$.  Because the array starts in the upper left and the
  kernel starts in the lower left, this mapping will flip vertically
  the visual ordering. The matrix array is in row-major form, so to
  get the column major form, the matrix must be transposed. To go from
  the mathematical representation, to the OpenVG input array
  representation, the steps must be done in reverse order. Thus, the
  transpose must be done before flipping.


\[
%%% trn([[3, 2, 1], [6, 5, 4], [9, 8, 7]] * [[0, 0, 1], [0, 1, 0], [1, 0, 0]]) => [[1, 4, 7], [2, 5, 8], [3, 6, 9]] %%%
 \left( \begin{pmatrix} 3 & 2 & 1 \\ 6 & 5 & 4 \\ 9 & 8 & 7 \end{pmatrix}
  \begin{pmatrix} 0 & 0 &
 1 \\ 0 & 1 &
      0 \\ 1 & 0 &
           0 \end{pmatrix} \right)^T \to \begin{pmatrix} 1 & 4 & 7 \\ 2 & 5 & 8 \\ 3 & 6 & 9 \end{pmatrix}
\]
\end{proof}
\subsection{Value Mapping}
\label{sec-3-1-5}
OpenVG has two functions, {\tt vgLookup} and {\tt vgLookupSingle},
that can be used for function mapping. The mapping functions are
represented by arrays of length 256, with values ranging from 0 to
255.  {\tt vgLookup} uses a different mapping function for each
channel.

\[\begin{pmatrix} R \\ G \\ B \\ A \end{pmatrix} \begin{matrix}
    f_R\atop\longrightarrow \\ f_G\atop\longrightarrow \\ f_B\atop\longrightarrow \\ f_A\atop\longrightarrow
    \end{matrix} \begin{pmatrix} R' \\ G' \\ B' \\ A' \end{pmatrix} \]
While, {\tt vgLookupSingle} maps a single channel, $C$, to a value
representing all four channels using a single mapping function \cite{openvg}.
\[C\rightarrow\begin{pmatrix} R' \\ G' \\ B' \\ A' \end{pmatrix}
    \]

CVPI provides a number of function arrays for use with {\tt vgLookup}.
\begin{description}
\item[{cvpi\_identity\_array}] The input values equal the output values.
\item[{cvpi\_inversion\_array}] Output equals $255 - \text{input}$.
\item[{cvpi\_255\_array}] All values are mapped to 255.
\item[{cvpi\_zeros\_array}] All values are mapped to 0.
\item[{cvpi\_binary\_array}] 0 is mapped to 0, and non-zero values are mapped to 255.
\item[{cvpi\_binary\_array\_inverted}] 0 is mapped to 255, and non-zero values mapped to 0.
\item[{cvpi\_sqrt\_array\_floor}] $\left\lfloor\sqrt{x}\right\rfloor$
\item[{cvpi\_sqrt\_array\_ceil}] $\left\lceil\sqrt{x}\,\right\rceil$
\item[{cvpi\_sqrt\_array\_round}] $\left\lfloor\sqrt{x}+0.5\right\rfloor$
\end{description}

\section{Filters}
\label{sec-3-2}
CVPI provides a number of pre-defined feature detection
kernels. It also provides a function, {\tt cvpi\_image\_magnitude}, for
calculating the gradient magnitude, that is

\[\sqrt{K_1^2+K_2^2}\]

where $K_1$ and $K_2$ are the convolution results. The function is
applied independently to each channel.

\begin{figure}[H]
\begin{mdframed}[style=default]
\begin{tabular}{rcc}
& Y & X \\
Sobel &
$\begin{pmatrix} -1 & -2 & -1 \\ 0 & 0 & 0 \\ 1 & 2 & 1 \end{pmatrix}$&
$\begin{pmatrix} -1 & 0 & 1 \\ -2 & 0 & 2 \\ -1 & 0 & 1 \end{pmatrix}$\\
Scharr &
$\begin{pmatrix} 3 & 10 & 3 \\ 0 & 0 & 0 \\ -3 & -10 & -3 \end{pmatrix}$&
$\begin{pmatrix} 3 & 0 & -3 \\ 10 & 0 & -10 \\ 3 & 0 & -3 \end{pmatrix}$\\
Prewitt&
$\begin{pmatrix} -1 & -1 & -1 \\ 0 & 0 & 0 \\ 1 & 1 & 1 \end{pmatrix}$&
$\begin{pmatrix} -1 & 0 & 1 \\ -1 & 0 & 1 \\ -1 & 0 & 1 \end{pmatrix}$\\
Robert's Cross &
$\begin{pmatrix} 0 & 1 \\ -1 & 0 \end{pmatrix}$&
$\begin{pmatrix} 1 & 0 \\ 0 & -1 \end{pmatrix}$\\
\end{tabular}
\end{mdframed}
  \caption{Predefined convolution matrices \cite{HIPR2}\cite{wiki:Sobel}.}
\end{figure}
\subsection{Magnitude}
\label{sec-3-2-1}
\[G = \sqrt{G_x^2+ G_y^2}\].

The function {\tt cvpi\_image\_magnitude} calculates the pixel
magnitude between two images. For the magnitude to be computed on the
GPU, the sum cannot exceed 255. The inputs $G_x$ and $G_y$ are integer
values between 0 and 255. Let $G_x = ax$ and $G_y = ay$, the formula
becomes
\[G = a\sqrt{x^2+ y^2}\],
where $x^2+y^2 \le 255$, $x^2\le 127.5$,
and $y^2\le 127.5$.
Given $G_x^2 = a^2x^2$ and solving for $a$ where $G_x=255$ and
$x=\sqrt{127.5}$, $a = 22.5831795814$.

Three {\tt vgLookup} array functions were calculated with $G_x$ as the
index input.

\[x_{\text{ceil}} = \left\lceil\frac{G_x}{a}\right\rceil,\; x_{\text{floor}} =
  \left\lfloor\frac{G_x}{a}\right\rfloor,\text{ and } x_{\text{round}} = \left\lfloor0.5+\frac{G_x}{a}\right\rfloor\]

{\tt cvpi\_image\_magnitude} applies the same look-up table to both inputs,
$G_x$ and $G_y$. The user specifies which table to use. The
resulting images are added together with {\tt cvpi\_image\_add}. The sums
will vary between 0 and 255. The square root function is applied to
the sum, $s$, using {\tt vgLookup}. Which of the three square root
functions to apply depends on the user.

\[\left\lceil\sqrt{s}\right\rceil,\left\lfloor\sqrt{s}\right\rfloor,\left\lfloor0.5+\sqrt{s}\right\rfloor\]

So far $\sqrt{x^2+ y^2}$ has been calculated. So to get the
magnitude $G$, the image channels are multiplied by $a$, using
{\tt vgColorMatrix}.
\section{OpenVG API Wrappers}
\label{sec-3-3}
There are a number of convenience functions in {\tt cvpi\_vg\_ext.h} that
wrap OpenVG functions; they do not offer additional functionality
but may be more intuitive and cut down on code size.

The functions {\tt vgConvolve} and {\tt vgSeparableConvolve} scale
image values between 0 and 1. It would be more intuitive and
convenient if the values ranged from 0 to 255. {\tt vgConvolveNormal}
and {\tt vgSeparableConvolveNormal} take the same parameters as {\tt
  vgConvolve} and {\tt vgSeparableConvolve}, but the bias
parameter is divided by 255 before being passed to the wrapped
function.

Another problem with {\tt vgConvolve} and {\tt vgSeparableConvolve} is
that the pixel being convolved over is in the lower left hand corner
of the kernel instead of in the center. The functions {\tt
  vgConvolveNoShift}, {\tt vgSeparableConvolveNoShift}, {\tt
  vgConvolveNormalNoShift}, and {\tt vgSeparableConvolveNormalNoShift}
move the convolved pixel to the center of the kernel by shifting in
the $y$ up by $\left\lceil\frac{h}{2}\right\rceil$ and shifting in the
$x$ right by $\left\lceil\frac{w}{2}\right\rceil$. If the correction is not
done and zeros are given for the shifts, then data points in the
resulting image will be shifted towards the image origin.

The function {\tt vgColorMatrix} also scales image values between 0
and 1.\\ {\tt vgColorMatrixNormal} allows the user to specify values
between 0 and 255 for the bias.

The function {\tt vgCreateImagePainted} acts like {\tt vgCreateImage} but
adds four color parameters, allowing the user to create a new
image having a solid color other than the default white.

The function {\tt vgPixelBits} takes a VGImageFormat and returns the
number of bits per pixel for that format.

\subsubsection{Emacs Calc Extension}
\label{sec-3-1-4-1}
In the {\tt extras} directory, there is the file {\tt convolve.el}. It
contains Emacs-Lisp Calc code to calculate convolutions, with the
convolution kernel origin at the kernel's center pixel. The first
argument is the image, the second the kernel, and the third is
optional. If the third argument is omitted, then the image will be
padded with the same value as the closest edge value. If a number is
given, then the input matrix will be padded with the given value. This
is useful for investigating how convolutions behave.

\begin{lstlisting}
convolve([[0,0,0,0,0],
          [0,1,1,1,0],
          [0,1,1,1,0],
          [0,1,1,1,0],
          [0,0,0,0,0]],
         [[1,1,1],[1,1,1],[1,1,1]])
\end{lstlisting}

     \[
%%% convolve([[0, 0, 0, 0, 0], [0, 1, 1, 1, 0], [0, 1, 1, 1, 0], [0, 1, 1, 1, 0], [0, 0, 0, 0, 0]], [[1, 1, 1], [1, 1, 1], [1, 1, 1]]) => [[1, 2, 3, 2, 1], [2, 4, 6, 4, 2], [3, 6, 9, 6, 3], [2, 4, 6, 4, 2], [1, 2, 3, 2, 1]] %%%
 convolve\left( \begin{pmatrix} 0 & 0 & 0 & 0 &
 0 \\ 0 & 1 & 1 & 1 & 0 \\ 0 & 1 & 1 & 1 & 0 \\ 0 & 1 & 1 & 1 & 0 \\ 0 & 0 & 0 & 0 & 0 \end{pmatrix},
\begin{pmatrix} 1 & 1 &
1 \\ 1 & 1 &
     1 \\ 1 & 1 &
          1 \end{pmatrix} \right) \to \begin{pmatrix} 1 & 2 & 3 & 2 &
1 \\ 2 & 4 & 6 & 4 & 2 \\ 3 & 6 & 9 & 6 & 3 \\ 2 & 4 & 6 & 4 & 2 \\ 1 & 2 & 3 & 2 & 1 \end{pmatrix}    \]

     \[
%%% convolve([[0, 0, 0, 0, 0], [0, 1, 1, 1, 0], [0, 1, 1, 1, 0], [0, 1, 1, 1, 0], [0, 0, 0, 0, 0]], [[1, 1, 1], [1, 1, 1], [1, 1, 1]], 1) => [[6, 5, 6, 5, 6], [5, 4, 6, 4, 5], [6, 6, 9, 6, 6], [5, 4, 6, 4, 5], [6, 5, 6, 5, 6]] %%%
 convolve\left( \begin{pmatrix} 0 & 0 & 0 & 0 &
 0 \\ 0 & 1 & 1 & 1 & 0 \\ 0 & 1 & 1 & 1 & 0 \\ 0 & 1 & 1 & 1 & 0 \\ 0 & 0 & 0 & 0 & 0 \end{pmatrix},
\begin{pmatrix} 1 & 1 & 1 \\ 1 & 1 & 1 \\ 1 & 1 & 1 \end{pmatrix},
1 \right) \to \begin{pmatrix} 6 & 5 & 6 & 5 &
6 \\ 5 & 4 & 6 & 4 & 5 \\ 6 & 6 & 9 & 6 & 6 \\ 5 & 4 & 6 & 4 & 5 \\ 6 & 5 & 6 & 5 & 6 \end{pmatrix}
   \]


\section{YUYV to YUVA}
\label{sec-3-4}
Unlike OpenCV, CVPI does not natively support camera capture; however,
this can be achieved using a video capture API, such as Video4Linux
(V4L). Most low-cost web cameras output to YUYV, where two horizontal
adjacent pixels are represented by 4 bytes. Both pixels share the U
and V channels but have different Y channels \cite{videoForLinux}. The
function {\tt cvpi\_yuyv2yuva} splits the pixels into separate 4 byte
blocks. This function allows the other OpenVG and CVPI functions to
manipulate camera input; all other functions require that each pixel
have its own separate 4 byte representation. The user must read the
raw image data from memory into a VGImage, using CVPI's default pixel
format VG\_sARGB\_8888, and pass it to {\tt cvpi\_yuyv2yuva}. The
output image will be the same height, but twice the width. The alpha
channel will be set to 255.

The input image channel mapping is:
 \[\begin{pmatrix} Y2 \\ U \\ Y1 \\ V \end{pmatrix}
 \rightarrow
\begin{pmatrix} R \\ G \\ B \\ A \end{pmatrix} \]
where, $Y1$ is the yellow channel for the first pixel and $Y2$
is the yellow channel for the second pixel.

Using the function {\tt vgColorMatrixNormal}, the input image is split into
two different images with the channels reordered and the alpha
channel set to the maximum value, 255.

\[I_1 \gets \begin{pmatrix} 0 & 0 & 0 & 1 \\ 0 & 1 & 0 & 0 \\ 0 & 0 & 1 & 0 \\ 1 & 0 & 0 & 0 \end{pmatrix}
\begin{pmatrix} Y2 \\ U \\ Y1 \\ V \end{pmatrix} + \begin{pmatrix}
0 \\ 0 \\ 0 \\ 255 \end{pmatrix} =\begin{pmatrix} V \\ U \\ Y1 \\ 255 \end{pmatrix}\to \begin{pmatrix} R \\ G \\ B \\ A \end{pmatrix}
\]

\[I_2 \gets \begin{pmatrix} 0 & 0 & 0 & 1 \\ 0 & 1 & 0 & 0 \\ 1 & 0 & 0 & 0 \\ 0 & 0 & 1 & 0 \end{pmatrix}
\begin{pmatrix} Y2 \\ U \\ Y1 \\ V \end{pmatrix}  + \begin{pmatrix}
0 \\ 0 \\ 0 \\ 255 \end{pmatrix} = \begin{pmatrix}
V \\ U \\ Y2 \\ 255 \end{pmatrix} \to \begin{pmatrix} R \\ G \\ B \\ A \end{pmatrix}
\]

The two images, $I_1$ and $I_2$, are then combined in vertical,
1-pixel wide strips using {\tt vgCopyImage}. If the image is written
to memory, $Y$ will be at array index 0, $U$ at index 1, $V$ at index
2, and $A$ at index 3.

\section{Color to Black and White}
\label{sec-3-5}
OpenVG can output to several formats including binary black and
white, one bit per pixel.  {\tt cvpi\_image\_rgba2bw} converts an
image channel to such an image. With {\tt vgLookup}, the
{\tt cvpi\_binary\_array} or {\tt cvpi\_binary\_array\_inverted} is used to
convert the desired channel to black and white, and
{\tt cvpi\_zeros\_array} is used to convert the other channels to
zero. {\tt vgCopyImage} is used to convert {\tt vgLookup}'s output to
a binary image; 255 gets mapped to 1 bits and 0 gets mapped to 0 bits.

\section{Image Addition and Subtraction}
\label{sec-3-6}
The function {\tt cvpi\_image\_add} allows the user to add and subtract
images of same dimensions. The function is described by the
formula:

\[C_{i,j} = s\cdot(a\cdot A_{i,j} + b\cdot B_{i,j})+t\]

where pixel values range between 0 and 255, $s$ is a floating point
value, $t$ is a bias passed to {\tt vgConvolveNormal}, and $a$ and $b$
are integers. $A$, $B$, and $C$ are images, with $i$ and $j$ being
pixel locations. The function can perform subtraction by setting $a$
or $b$ to a negative value.

If the image height is less than or equal to half the maximum height,
then the two images are combined into a single image by copying
horizontal strips, 1-pixel in height into another image of twice the
height, such that the first row is from the first image, the second
row is from the second image, and this is repeated for all following
rows. The resulting image is then convolved with {\tt
  vgConvolveNormal} using a $2\byX 1$ kernel, $k$, where
$k_{0,1} = a$ and $k_{0,0} = b$.

OpenVG's convolution formula becomes:
\[I(x,y) = s\left(\sum_{j=0}^1 k(0,1-j)\cdot P(x,y+j)\right)+t\]
\[I(x,y) = s(k_{0,1}\cdot P(x,y) + k_{0,0}\cdot P(x,y+1))+t\]
\[I(x,y) = s(a\cdot P(x,y) + b\cdot P(x,y+1))+t\]
where $P$ is the input image and $I$ is the output image, with
pixel coordinate parameters, the origin being in the lower left
corner.

{\tt vgCopyImage} is used to copy the even-indexed rows to another
image, and the odd indexed rows are thrown out.

For example, if two images $A$ and $B$, both 1-pixel wide and two
pixels high, are to be added, the following steps are performed.
\begin{figure}[H]
\begin{mdframed}[style=default]
\begin{enumerate}
\item Combining the images:
\begin{center}
\begin{tabular}{cl}
Index &\ \\
\hline
3 & B10\\
2 & A10\\
1 & B00\\
0 & A00\\
\end{tabular}
\end{center}

\item Convolution:
\begin{center}
\begin{tabular}{cl}
Index &\ \\
\hline
3 & B10+Padding\\
2 & A10+B10\\
1 & B00+A10\\
0 & A00+B00\\
\end{tabular}
\end{center}

\item Removing the odd numbered rows:
\begin{center}
\begin{tabular}{cl}
Index &\ \\
\hline
2 & A10+B10\\
0 & A00+B00\\
\end{tabular}
\end{center}
\end{enumerate}
\end{mdframed}
\caption{Example: Adding two $1\byX 2$ images, A and B.}
\end{figure}

For images larger in height than half the max height, the images
are split in half, and their upper and lower halves are added
separately. If the image height is odd, then the top rows are
added separately from the other rows. That is, the image minus the
top row is split in half and added like in the even case, and then
the top rows of each image are added together.
\begin{figure}[H]
\begin{mdframed}[style=default]
\begin{enumerate}
\item Two images are created, combining image rows:
\begin{center}
\begin{tabular}{cl}
\multicolumn{2}{c}{Upper}\\
Index &\ \\
\hline
3 & B30\\
2 & A30\\
1 & B20\\
0 & A20\\
\end{tabular}
\end{center}
\begin{center}
\begin{tabular}{cl}
\multicolumn{2}{c}{Lower}\\
Index &\ \\
\hline
3 & B10\\
2 & A10\\
1 & B00\\
0 & A00\\
\end{tabular}
\end{center}

\item Convolution:
\begin{center}
\begin{tabular}{cl}
\multicolumn{2}{c}{Upper}\\
Index &\ \\
\hline
3 & B30+Padding\\
2 & A30+B30\\
1 & B20+A30\\
0 & A20+B20\\
\end{tabular}
\end{center}
\begin{center}
\begin{tabular}{cl}
\multicolumn{2}{c}{Lower}\\
Index &\ \\
\hline
3 & B10+Padding\\
2 & A10+B10\\
1 & B00+A10\\
0 & A00+B00\\
\end{tabular}
\end{center}
\item Removing the odd numbered rows and combining into a single image:
\begin{center}
\begin{tabular}{cl}
Index &\ \\
\hline
2 & A30+B30\\
0 & A20+B20\\
2 & A10+B10\\
0 & A00+B00\\
\end{tabular}
\end{center}
\end{enumerate}
\end{mdframed}
\caption{Example: Adding two $1\byX 4$ images, A and B, where the
  maximum allowed image height is 4.}
\end{figure}
\begin{figure}[H]
\begin{mdframed}[style=default]
\begin{enumerate}
\item Three images are created.
\begin{center}
\begin{tabular}{cl}
\multicolumn{2}{c}{Top-row}\\
Index &\ \\
\hline
1 & B40\\
0 & A40\\
\end{tabular}
\end{center}
\begin{center}
\begin{tabular}{cl}
\multicolumn{2}{c}{Upper}\\
Index &\ \\
\hline
3 & B30\\
2 & A30\\
1 & B20\\
0 & A20\\
\end{tabular}
\end{center}
\begin{center}
\begin{tabular}{cl}
\multicolumn{2}{c}{Lower}\\
Index &\ \\
\hline
3 & B10\\
2 & A10\\
1 & B00\\
0 & A00\\
\end{tabular}
\end{center}

\item Convolution:
\begin{center}
\begin{tabular}{cl}
\multicolumn{2}{c}{Top-row}\\
Index &\ \\
\hline
1 & B40+Padding\\
0 & A40+B40\\
\end{tabular}
\end{center}
\begin{center}
\begin{tabular}{cl}
\multicolumn{2}{c}{Upper}\\
Index &\ \\
\hline
3 & B30+Padding\\
2 & A30+B30\\
1 & B20+A30\\
0 & A20+B20\\
\end{tabular}
\end{center}
\begin{center}
\begin{tabular}{cl}
\multicolumn{2}{c}{Lower}\\
Index &\ \\
\hline
3 & B10+Padding\\
2 & A10+B10\\
1 & B00+A10\\
0 & A00+B00\\
\end{tabular}
\end{center}
\end{enumerate}
\end{mdframed}
\caption{Example: Adding two $1\byX 5$ images, A and B, where the
  maximum allowed image height is 5. \emph{(cont.)}}
\end{figure}
\pagebreak
\begin{figure}[H]
\ContinuedFloat
\begin{mdframed}[style=default]
\begin{enumerate}
\setcounter{enumi}{2}
\item Removing the odd numbered rows and combining into a single image:
\begin{center}
\begin{tabular}{cl}
Index &\ \\
\hline
0 & A40+B40\\
2 & A30+B30\\
0 & A20+B20\\
2 & A10+B10\\
0 & A00+B00\\
\end{tabular}
\end{center}
\end{enumerate}
\end{mdframed}
\caption{Example: Adding two $1\byX 5$ images, A and B, where the
  maximum allowed image height is 5.}
\end{figure}
\section{Channel Addition and Subtraction}
\label{sec-3-7}
There are three functions for adding pixel channels of the same
image together; {\tt cvpi\_channel\_add},
{\tt cvpi\_color\_channels\_add}, and {\tt cvpi\_all\_channels\_add}.
These functions use {\tt vgColorMatrixNormal} to perform the adding.

The function, {\tt cvpi\_all\_channels\_add}, allows the user to specify a
scalar for each channel, which channels to output to, and separate biases
for each output channel.
   \[
%%% [[R'], [G'], [B'], [A']] := [[S_R, S_G, S_B, S_A], [0, 1, 0, 0], [0, 0, 1, 0], [0, 0, 0, 1]] * [[R], [G], [B], [A]] + [[B_R], [0], [0], [0]] => [[(R * S_R + G * S_G + B * S_B + A * S_A + B_R)], [G], [B], [A]] %%%
 \begin{pmatrix} R' \\ G' \\ B' \\ A' \end{pmatrix}
          \gets \begin{pmatrix} S_R & S_G & S_B &
 S_A \\ 0 & 1 & 0 & 0 \\ 0 & 0 & 1 & 0 \\ 0 & 0 & 0 & 1 \end{pmatrix}
  \begin{pmatrix} R \\ G \\ B \\ A \end{pmatrix}
  + \begin{pmatrix} B_R \\ 0 \\ 0 \\ 0 \end{pmatrix} \to \begin{pmatrix} R S_R + G S_G + B S_B + A S_A
  + B_R \\ G \\ B \\ A \end{pmatrix}
\]
\[
%%% [[R'], [G'], [B'], [A']] := [[S_R, S_G, S_B, S_A], [0, 1, 0, 0], [S_R, S_G, S_B, S_A], [0, 0, 0, 1]] * [[R], [G], [B], [A]] + [[B_R], [0], [B_B], [0]] => [[(R * S_R + G * S_G + B * S_B + A * S_A + B_R)], [G], [(R * S_R + G * S_G + B * S_B + A * S_A + B_B)], [A]] %%%
 \begin{pmatrix} R' \\ G' \\ B' \\ A' \end{pmatrix}
          \gets \begin{pmatrix} S_R & S_G & S_B &
 S_A \\ 0 & 1 & 0 & 0 \\ S_R & S_G & S_B & S_A \\ 0 & 0 & 0 & 1 \end{pmatrix}
  \begin{pmatrix} R \\ G \\ B \\ A \end{pmatrix}
  + \begin{pmatrix} B_R \\ 0 \\ B_B \\ 0 \end{pmatrix} \to \begin{pmatrix} R S_R + G S_G + B S_B + A S_A
  + B_R \\ G \\ R S_R + G S_G + B S_B + A S_A + B_B \\ A \end{pmatrix}
\]
The functionality is similar for {\tt cvpi\_color\_channels\_add}.
   \[
%%% [[R'], [G'], [B'], [A']] := [[1, 0, 0, 0], [0, 1, 0, 0], [0, 0, 1, 0], [S_R, S_G, S_B, 0]] * [[R], [G], [B], [A]] + [[0], [0], [0], [B_A]] => [[R], [G], [B], [(R * S_R + G * S_G + B * S_B + B_A)]] %%%
 \begin{pmatrix} R' \\ G' \\ B' \\ A' \end{pmatrix}
          \gets \begin{pmatrix} 1 & 0 & 0 &
 0 \\ 0 & 1 & 0 & 0 \\ 0 & 0 & 1 & 0 \\ S_R & S_G & S_B & 0 \end{pmatrix}
  \begin{pmatrix} R \\ G \\ B \\ A \end{pmatrix}
  + \begin{pmatrix} 0 \\ 0 \\ 0 \\ B_A \end{pmatrix} \to \begin{pmatrix} R \\ G \\ B \\ R S_R + G S_G
  + B S_B + B_A \end{pmatrix}
\]
   \[
%%% [[R'], [G'], [B'], [A']] := [[S_R, S_G, S_B, 0], [0, 1, 0, 0], [0, 0, 1, 0], [0, 0, 0, 1]] * [[R], [G], [B], [A]] + [[B_R], [0], [0], [0]] => [[(R * S_R + G * S_G + B * S_B + B_R)], [G], [B], [A]] %%%
 \begin{pmatrix} R' \\ G' \\ B' \\ A' \end{pmatrix}
          \gets \begin{pmatrix} S_R & S_G & S_B &
 0 \\ 0 & 1 & 0 & 0 \\ 0 & 0 & 1 & 0 \\ 0 & 0 & 0 & 1 \end{pmatrix}
  \begin{pmatrix} R \\ G \\ B \\ A \end{pmatrix}
  + \begin{pmatrix} B_R \\ 0 \\ 0 \\ 0 \end{pmatrix} \to \begin{pmatrix} R S_R + G S_G + B S_B
  + B_R \\ G \\ B \\ A \end{pmatrix}
\]

With the function {\tt cvpi\_channel\_add}, the user selects which two
channels to add and the output channels. The effective adding formula
is the same as that used by {\tt cvpi\_image\_add}, but is performed
by {\tt vgColorMatrixNormal}. If the two input channels are the same,
then their scalars are added together.  For example, adding channel R
with channel G and outputting to B and A, with B's bias = 0 and A's
bias = 1.
   \[
%%% [[1, 0, 0, 0], [0, 1, 0, 0], [1, 1, 0, 0], [1, 1, 0, 0]] * [[1], [5], [10], [20]] + [[0], [0], [0], [1]] => [[1], [5], [6], [7]] %%%
 \begin{pmatrix} 1 & 0 & 0 & 0 \\ 0 & 1 & 0 & 0 \\ 1 & 1 & 0 & 0 \\ 1 & 1 & 0 & 0 \end{pmatrix}
          \begin{pmatrix} 1 \\ 5 \\ 10 \\ 20 \end{pmatrix}
          + \begin{pmatrix} 0 \\ 0 \\ 0 \\ 1 \end{pmatrix} \to \begin{pmatrix} 1 \\ 5 \\ 6 \\ 7 \end{pmatrix}
\]

\section{Combining Images by Channel}
\label{sec-3-8}
The function {\tt cvpi\_image\_combine\_channelwise} can combine two
images, where the user selects the channels to include from one image,
and then the complement of the selected channels are included from the
other image in the output. The function works by using {\tt
  vgColorMatrix} to produce two new images where excluded channels are
zeroed. Then the two resulting images are added using {\tt
  cvpi\_image\_add}.
For example, given two one pixel images, with the first two
channels from the first image and the second two channels in the
second image are selected.
   \[
%%% [[1, 0, 0, 0], [0, 1, 0, 0], [0, 0, 0, 0], [0, 0, 0, 0]] * [[1], [2], [3], [4]] + [[0, 0, 0, 0], [0, 0, 0, 0], [0, 0, 1, 0], [0, 0, 0, 1]] * [[5], [6], [7], [8]] => [[1], [2], [7], [8]] %%%
 \begin{pmatrix} 1 & 0 & 0 & 0 \\ 0 & 1 & 0 & 0 \\ 0 & 0 & 0 & 0 \\ 0 & 0 & 0 & 0 \end{pmatrix}
          \begin{pmatrix} 1 \\ 2 \\ 3 \\ 4 \end{pmatrix}
          + \begin{pmatrix} 0 & 0 & 0 &
 0 \\ 0 & 0 & 0 & 0 \\ 0 & 0 & 1 & 0 \\ 0 & 0 & 0 & 1 \end{pmatrix}
  \begin{pmatrix} 5 \\ 6 \\ 7 \\ 8 \end{pmatrix} \to \begin{pmatrix} 1 \\ 2 \\ 7 \\ 8 \end{pmatrix}
\]

\section{Thresholding}
\label{sec-3-9}
CVPI contains five thresholding functions;
{\tt cvpi\_channel\_threshold}, {\tt cvpi\_image\_threshold},
{\tt cvpi\_channel\_threshold\_sector}, {\tt cvpi\_image\_threshold\_sector}, and
{\tt cvpi\_image\_threshold\_adaptive\_mean}. Thresholding is used to
filter out information within a certain value range.

For {\tt cvpi\_channel\_threshold} and {\tt cvpi\_image\_threshold},
the user specifies value ranges, whether to keep or remove pixels
whose values are within the range while doing the opposite for those
outside the range, and what the replacing value will be for removed
pixels. {\tt cvpi\_image\_threshold} adds the additional option to
specify whether channels are dependent or independent. That is, if any
channel in a pixel falls outside the threshold range, and if the
channels are dependent, then the pixel's channels will be removed. If they
are independent, then a pixel channel is kept or removed regardless of
whether the other channels were within the threshold range.

The {\tt cvpi\_channel\_threshold} function works by building an array to
pass to {\tt vgLookup}. For channels that are not thresholded, their
{\tt vgLookup} functions are set to the identity function array.

The functions {\tt cvpi\_channel\_threshold\_sector} and {\tt
  cvpi\_image\_threshold\_sector} threshold a channel or image in
sectors or blocks. The user must pass in a function that is used to
compute a statistic, such as the average, about the sector and the
statistic is used as the upper bound for masking or not masking,
depending on what the user specifies. The lower bound is zero. The
image is divided into {\tt vgChildImage} sectors. If the height or
width does not divide evenly, then their are additional sectors around
the upper and right edges, whose width or height are the modulo of the
image and sector dimensions.

The function {\tt cvpi\_image\_threshold\_adaptive\_mean} adaptively
thresholds an image by using {\tt vgConvolveNormalNoShift} to find the
local mean for each pixel. The user specifies the $N\byX N$
kernel size. Each kernel element has the value 1. The scale is
$\left(\frac{1}{N}\right)^2$.  If the user wants to keep pixels
greater than the convolution results, then the resulting image from the
convolution gets subtracted from the original image. Else, if the
user wants to keep pixels less than the convolution results, the
original image gets subtracted from the image resulting from the
convolution. The user can also specify an integer bias, which is
passed to {\tt cvpi\_image\_add} as the bias parameter when finding the
difference between the images. Image values are saturated between 0
and 255, so negative values become zero and values above 255
become 255. The image resulting from the difference can be used as
a mask on the original image; values that are zero are pixels that
are to be removed by the threshold. To create the masking image,
values in the difference image are mapped to 0 if they are
non-zero, and mapped to 255 if they are zero. If the user wants
removed pixels to be shown in white, then the mask is added to the
original image; else if the user wants removed pixels to be shown
in black, then the mask is subtracted from the original image.
If the user wants the image channels to be dependent on each-other;
that is, only keep those pixels where all of the channels survive
thresholding, then the mean image is split into four separate
images, one for each color channel. The resulting images are then
AND'ed together (see section \ref{sec-3-12}) and then turned into the
masking image.

\section{Masking}
\label{sec-3-10}
Masking one image with another can be accomplished easily with
{\tt cvpi\_image\_add}, by setting the masking pixels to 255 and the
non-masking pixels to 0, and then subtracting it from or adding it
to the image being masked. However, the user might want to keep
masking information inside an image channel, such as the alpha
channel. The function, {\tt cvpi\_image\_mask\_channel}, adds or subtracts
a channel from the other image channels using {\tt vgColorMatrix}. It
is assumed that the masking channel is already set to the correct
values.

If alpha is the masking channel and the mask is being added, then:
\[
\begin{pmatrix} 1 & 0 & 0 & 1 \\ 0 & 1 & 0 & 1 \\ 0 & 0 & 1 & 1 \\ 0 & 0 & 0 & 1 \end{pmatrix}
  \begin{pmatrix} R \\ G \\ B \\ A \end{pmatrix} \to \begin{pmatrix} R + A \\ G + A \\ B + A \\ A \end{pmatrix}
\]
   If red is the masking channel and the mask is being subtracted, then:
   \[\begin{pmatrix} 1 & 0 & 0 & 0 \\ -1 & 1 & 0 & 0 \\ -1 & 0 & 1 & 0 \\ -1 & 0 & 0 & 1 \end{pmatrix}\begin{pmatrix} R \\ G \\ B \\ A \end{pmatrix} \to \begin{pmatrix} R \\ G - R \\ B - R \\ A - R \end{pmatrix}
\]

\section{Statistics}
\label{sec-3-11}
CVPI contains some statistical operations. Statistical operations,
being reduction type operations, cannot really utilize the
GPU. However, some mapping functions, such as thresholding, require
results from statistical operations.
\subsection{Average}
\label{sec-3-11-1}
CVPI provides two averaging functions; {\tt
  cvpi\_image\_mean\_arithmetic} and {\tt cvpi\_image\_mean\_gpu}.
{\tt cvpi\_image\_mean\_arithmetic} computes the arithmetic mean for
each channel. This operation is performed entirely on the CPU. The
function is not heavily optimized and could be improved on ARM CPUs
supporting NEON SIMD. {\tt cvpi\_image\_mean\_gpu} uses convolution in
a way similar to {\tt cvpi\_image\_add} by averaging vertically
adjacent pixels until the resulting image is one pixel high. The image
is reduced in height by 1 for each iteration. A one-pixel image is
produced by averaging horizontally adjacent pixels in the one-pixel
high image.  The resulting value is not a true statistic but is close
to the average. The exact average is not always desired in computer
vision and something like it may be good enough. The function has a
parameter to limit the number of iterations, returning an image with
dimensions: (input width) $\times$ ((height -- number of iterations)
or (1, if height -- number of iterations $\le 0$)).

\subsection{Channel to Data}
\label{sec-3-11-2}
The function {\tt vgGetImageSubData}, returns the data for an entire
image \cite{openvg}. {\tt cvpi\_channel2data} returns the data for
just a single channel. The function works by reordering the channels
so that the desired data is in the alpha channel using {\tt
  vgColorMatrix}. The image is then copied to an image of type
VG\_A\_8. This will discard the non-alpha channels, and then the alpha
image is written to memory.

    \[
%%% [[0, 0, 0, 0], [0, 0, 0, 0], [0, 0, 0, 0], [1, 0, 0, 0]] * [[R], [G], [B], [A]] => [[0], [0], [0], [R]] %%%
 \begin{pmatrix} 0 & 0 & 0 & 0 \\ 0 & 0 & 0 & 0 \\ 0 & 0 & 0 & 0 \\ 1 & 0 & 0 & 0 \end{pmatrix}
          \begin{pmatrix} R \\ G \\ B \\ A \end{pmatrix} \to \begin{pmatrix} 0 \\ 0 \\ 0 \\ R \end{pmatrix}
\]

    \[
%%% [[0, 0, 0, 0], [0, 0, 0, 0], [0, 0, 0, 0], [0, 1, 0, 0]] * [[R], [G], [B], [A]] => [[0], [0], [0], [G]] %%%
 \begin{pmatrix} 0 & 0 & 0 & 0 \\ 0 & 0 & 0 & 0 \\ 0 & 0 & 0 & 0 \\ 0 & 1 & 0 & 0 \end{pmatrix}
          \begin{pmatrix} R \\ G \\ B \\ A \end{pmatrix} \to \begin{pmatrix} 0 \\ 0 \\ 0 \\ G \end{pmatrix}
\]
    \[
%%% [[0, 0, 0, 0], [0, 0, 0, 0], [0, 0, 0, 0], [0, 0, 1, 0]] * [[R], [G], [B], [A]] => [[0], [0], [0], [B]] %%%
 \begin{pmatrix} 0 & 0 & 0 & 0 \\ 0 & 0 & 0 & 0 \\ 0 & 0 & 0 & 0 \\ 0 & 0 & 1 & 0 \end{pmatrix}
          \begin{pmatrix} R \\ G \\ B \\ A \end{pmatrix} \to \begin{pmatrix} 0 \\ 0 \\ 0 \\ B \end{pmatrix}
\]

\subsection{Max and Min}
\label{sec-3-11-3}
CVPI has three functions for determining a channel's minimum and
maximum values; {\tt cvpi\_channel\_max}, {\tt cvpi\_channel\_min}, and
{\tt cvpi\_channel\_max\_min}. All three functions pass the input image
and channel specifier to {\tt cvpi\_channel2data}, and then use
sequential searches.

\subsection{Histogram and Cumulative Distribution}
\label{sec-3-11-4}
{\tt cvpi\_channel\_histogram}, {\tt cvpi\_color\_\-channels\_histogram}, and
{\tt cvpi\_image\_histogram} create arrays in memory where the index
represents an 8-bit color value, and the index entry represent the
number of times that value occurs in an image.
{\tt cvpi\_channel\_histogram}'s data array can represent one of four
channels: red, green, blue, or
alpha. {\tt cvpi\_color\_\-channels\_histogram}'s data is an array of 768
elements where the first 256 elements represent red values, then
next 256 elements represent green values, and the last 256
elements represent blue values. {\tt cvpi\_image\_histogram} adds
another 256 entries, after the blue entries, for the alpha channel
values. This way channel data is kept contiguous and the data
format is consistent between these functions.

The functions {\tt cvpi\_channel\_\-cumulative\_\-distribution}, {\tt
  cvpi\_color\_\-channels\_\-cumulative\_\-distribution}, and {\tt
  cvpi\_image\_\-cumulative\_distribution}, take the respective
outputs of {\tt cvpi\_channel\_histogram}, {\tt
  cvpi\_color\_\-channels\_histogram}, and {\tt
  cvpi\_image\_histogram}, and return the same type of data
structures, except that it is a cumulative distribution. This can be
used for histogram equalization. Each entry is computed using the
formula:

\[c(I) = \frac{1}{N}\sum_{i=0}^Ih(i)\]
where $h(i)$ is the value computed by the corresponding histogram
function \cite[p.~95]{Szeliski}.

These functions are used by {\tt cvpi\_channel\_\-histogram\_equalization},
{\tt cvpi\_color\_\-channels\_\-histogram\_equalization}, and
{\tt cvpi\_image\_\-histogram\_equalization} to create histogram equalized
images. First, the histogram $h(i)$, and then the cumulative
distribution $c(I)$ are computed. The output of the cumulative
distribution function is converted to a VGubyte array. If the user
wants the array values scaled from 0 to 255, then the $c(I)$ array
is re-scaled using the formula

\[h(v)=\mathrm{round}\left(\frac{cdf(v)- cdf_{min}}{(M\cdot N) - cdf_{min}}\cdot 255\right)\]

where $M$ and $N$ are the image dimensions, and $cdf_{min}$ is the
smallest positive value in the cumulative distribution
\cite{wiki:HistogramEqualization}. The $c(I)$ array is then passed to
{\tt vgLookup}. For channels not being operated on, the identity array
is given.

\section{Histogram Equalization}
CVPI has three histogram equalization functions; {\tt
  cvpi\_channel\_\-histogram\_equalization}, {\tt
  cvpi\_color\_\-channels\_\-histogram\_equalization}, and {\tt
  cvpi\_image\_\-histogram\_equalization}; and their sector
counterparts; {\tt cvpi\_channel\_\-histogram\_equalization\_sector},
{\tt cvpi\_color\_\-channels\_\-histogram\_equalization\_sector}, and
{\tt cvpi\_image\_\-histogram\_equalization\_sector}. The first three
functions use the same histogram and cumulative distribution statistic
for the entire image. The latter three functions partition the image,
similar to {\tt cvpi\_channel\_threshold\_sector} and {\tt
  cvpi\_image\_threshold\_sector}, and apply their image-wide
counterparts to the sub-images. Adaptive histogram equalization is
not implemented.

\section{Logic Operations}
\label{sec-3-12}
CVPI can perform all binary logical operations: AND, NAND, OR, NOR,
XOR, XNOR, complement ($\backslash$), inverse complement. The complement operation
is the same as $A\&!B$.
\begin{center}
\begin{tabular}{cccccccccc}
A & B &  AND & OR & $A\backslash B$ & XOR & $!(A\backslash B)$ & NOR & XNOR & NAND\\
\hline
0 & 0 &   0 & 0 & 0 & 0 &  1 & 1 &  1   & 1\\
0 & 1 &   0 & 1 & 0 & 1 &  1 & 0 &  0   & 1\\
1 & 0 &   0 & 1 & 1 & 1 &  0 & 0 &  0   & 1\\
1 & 1 &   1 & 1 & 0 & 0 &  1 & 0 &  1   & 0\\
\end{tabular}
\end{center}
For two inputs, there are 16 possible operations The operations
$B\backslash A$ and $!(B\backslash A)$ can be achieved by reversing
the inputs. Operations that result in A, B, {\bf 1}, or {\bf 0} are
tautological. The NOT operator is the same as applying the inversion
array with {\tt vgLookup}.

\begin{figure}[H]
\begin{mdframed}[style=default]
\begin{enumerate}
\item Allow the user to specify what values true and false outputs will
be mapped to; TC and FC. The user can specify, for the input
images, whether 0 maps to true and !0 maps to false, or the
opposite.
\item Set all of A's pixels to 1's (TRUE) and 0's (FALSE).
\item Set all of B's pixels to 2's (TRUE) and 0's (FALSE).
\item Add the images.
\item Where TC = TRUE and FC = FALSE, use {\tt vgLookup} to:
\begin{description}
\item[{AND (intersection)}] set all 3's to TC and non-3's to FC.
\item[{NAND}] set all 3's to FC and non-3's to TC.
\item[{OR (union)}] set all 0's to FC and non-0's to TC.
\item[{NOR}] set all 0's to TC and non-0's to FC.
\item[{XOR (symmetric difference)}] set all 1's and 2's to TC and the rest to FC.
\item[{XNOR}] set all 1's and 2's to FC and the rest to TC.
\item[{COMPLEMENT (relative)}] set all 1's to TC and the rest to FC.
\item[{INV. COMPLEMENT}] set all 1's to FC and the rest to TC.
\end{description}
\end{enumerate}
\end{mdframed}
\caption{Algorithm for performing binary logical operations on images.}
\end{figure}

\section{Morphology}
\label{sec-3-13}
CVPI provides five morphology functions; {\tt cvpi\_image\_dilate},
{\tt cvpi\_image\_\-erode}, {\tt cvpi\_image\_\-hit\_miss}, {\tt cvpi\_image\_thin}, and
{\tt cvpi\_image\_thicken}. These functions treat non-zero values as true
and zero values as false, or the opposite if the user specifies
it; so input images are essentially treated as binary images. Like
the logic functions, the output is an image consisting of two-value
channels, one color being mapped to zero valued elements and
another color being mapped to non-zero valued
elements.

Assuming that the user chooses non-zero values as true,
{\tt cvpi\_image\_dilate} causes non-zero regions to
expand. {\tt cvpi\_image\_erode} causes non-zero regions to retract;
i.e. causes zero regions to expand. {\tt cvpi\_image\_\-hit\_miss} detects
corners. {\tt cvpi\_image\_thin} and {\tt cvpi\_image\_thicken} are similar to
{\tt cvpi\_image\_erode} and {\tt cvpi\_image\_dilate} but rely on
{\tt cvpi\_image\_\-hit\_miss}.

For each of these functions, given an image of zeros and non-zeros,
if non-zero values are true, then non-zero values are mapped to 1
and zero is mapped to 0, else the opposite will be done. A
$3\byX 3$ binary convolution kernel is then mapped over the output \cite[Morphology]{HIPR2}.
\subsection{Dilation and Erosion}
\label{sec-3-13-1}
The convolution kernel for dilation and erosion is
\[%%% [[1, 1, 1], [1, 1, 1], [1, 1, 1]] %%%
\begin{pmatrix} 1 & 1 & 1 \\ 1 & 1 & 1 \\ 1 & 1 & 1 \end{pmatrix}\]

The input image is converted to ``binary'', and the kernel is applied
\cite[Morphology]{HIPR2}. True values are mapped to the true color,
and false values are mapped to the false color. {\tt
  cvpi\_image\_erode} uses {\tt cvpi\_image\_dilate} with the false
color and true color switched; and with non-zero equal to false and
zero equal to true if the user specified non-zero equal to true, else
the other way around if the user specified zero equal to true. Erosion
is essentially dilation with the binary input values flipped.

\subsection{Thinning, Thickening, and the Hit-and-Miss Transform}
\label{sec-3-13-2}
The hit-and-miss transform uses four different convolution
kernels:
\[\begin{pmatrix} 0 & 1 & 0 \\ 0 & 1 & 1 \\ 0 & 0 & 0\end{pmatrix},
\begin{pmatrix} 0 & 0 & 0 \\ 0 & 1 & 1 \\ 0 & 1 & 0\end{pmatrix},
\begin{pmatrix} 0 & 0 & 0 \\ 1 & 1 & 0 \\ 0 & 1 & 0\end{pmatrix},
\begin{pmatrix} 0 & 1 & 0 \\ 1 & 1 & 0 \\ 0 & 0 & 0\end{pmatrix}\] The
input image, after being converted to ``binary'', is convolved with
each of these kernels, returning four different images, and then the
outputs are or-ed together \cite[Morphology]{HIPR2}.

For thinning and thickening, a hit-and-miss transform is performed on
the input image. In the thinning function, the logical complement
function is taken between the original image and the image returned by
the hit-and-miss transform. In the thickening function, the logical OR
function is taken between the original image and the image returned by
the hit-and-miss transform \cite[Morphology]{HIPR2}.

\section{Data Points}
\label{sec-3-14}
Once the GPU has removed extraneous information from an image, the
remaining data likely needs to be processed by the CPU. Often it is
the remaining pixel coordinates and not the pixel values that are
desired. With the function {\tt cvpi\_image\_coordinate\_table},
CVPI can return a coordinate table for the CPU to operate on.  The
user can specify whether the origin is in the upper or
lower left corner. {\tt cvpi\_image\_coordinate\_table} outputs the
input image to memory and uses the CPU to create a table of non-zero
pixels. The user can specify which channels to check for non-zero
data, unspecified channels are ignored.  The function assumes that the
image format being used has its origin in the upper left corner, so to
switch the origin to the lower left corner, it subtracts the $y$
coordinate from the image height. The function assumes that every
pixel could be non-zero, so sufficient space is allocated on the
heap to have a coordinate entry for every image pixel. The coordinate
data is represented by a union of a 32 bit unsigned integer with a two
element array of 16 bit unsigned integers, with the first element
representing the $x$ coordinate and the second, the $y$
coordinate. {\tt cvpi\_image\_coordinate\_table} returns a structure
consisting of a pointer to an array of coordinates and the length the
array. If not all of the allocated memory is used then the memory
allocation is reduced to fit the coordinate table data.

\section{YUVA to RGBA}
\label{sec-3-15}
Currently, there are no image formats that support YUVA, so the
image must be converted for human viewing.  RGBA is commonly
supported. This conversion must be done on the CPU.
The function {\tt cvpi\_avuy2argb} converts an image in CPU memory to an
RGBA image in memory. The function {\tt cvpi\_image2argb} is a simple
wrapper around {\tt cvpi\_avuy2argb} that takes a VGImage, writes it
to memory, and calls {\tt cvpi\_avuy2argb} on it.
\begin{figure}[H]
\begin{mdframed}[style=default]
\begin{lstlisting}
C = Y - 16
D = U - 128
E = V - 128

R = clip((298 * C           + 409 * E + 128) >> 8)
G = clip((298 * C - 100 * D - 208 * E + 128) >> 8)
B = clip((298 * C + 516 * D           + 128) >> 8)
\end{lstlisting}
\end{mdframed}
\caption{Algorithm for converting YUV to RGB \cite{yuv2rgb}.}
\end{figure}
\chapter{Logging}
\label{sec-5}
OpenVG has error logging. Errors can be returned using {\tt
  vgGetError()}. However, it is not possible to create new VG errors,
so CVPI uses its own error logging facilities.

CVPI provides logging functions. The user can compile CVPI to write
logs synchronously with {\tt fprintf}, asynchronously using POSIX
threads, or not at all. The log writing location can be changed with
{\tt cvpi\_log\_file\_set} and returned using {\tt cvpi\_log\_file\_get}. The
default location is {\tt stderr}. If the user wants to log to a file,
then {\tt cvpi\_log\_file\_set} must be used at the beginning of the
program and {\tt cvpi\_log\_file\_unset} at the end of the program. CVPI
provides seven logging functions; {\tt cvpi\_log}, {\tt cvpi\_log\_1},
{\tt cvpi\_log\_2}, {\tt cvpi\_log\_3}, {\tt cvpi\_log\_4}, {\tt cvpi\_log\_5}, and
{\tt cvpi\_log\_6}, though {\tt cvpi\_log} is not intended for direct use. The
six numbered logging functions have corresponding formatting
functions; {\tt cvpi\_log\_format\_1}, {\tt cvpi\_log\_format\_2},
{\tt cvpi\_log\_format\_3}, {\tt cvpi\_log\_format\_4}, {\tt cvpi\_log\_format\_5}, and
{\tt cvpi\_log\_format\_6}. This is used to standardize log formatting and
allow different log strings to be passed to a POSIX thread when
performing asynchronous logging. POSIX threads can only take a
single argument that is normally a pointer. The logging thread is
passed a pointer to a structure containing the log format type,
which tells the thread what other structure parameters are to be
printed. The structure contains an {\tt fprintf} formatting string, plus
the arguments to the string. The log type tells the thread what
structure elements to pass to {\tt fprintf}. The logging thread is
detached after being created.

\chapter{Image Headers}
\label{sec-7}
CVPI has functions for writing image headers for Portable Bitmap
(PBM), Portable Graymap (PGM), and Bitmap (BMP). The PBM and PGM
functions, {\tt cvpi\_pbm\_header\_write} and {\tt
  cvpi\_pgm\_header\_write}, are wrappers around {\tt fprintf}, which
write to the specified file. The BMP functions, {\tt
  cvpi\_bmp\_header\_alloc}, {\tt cvpi\_bmp\_header\_write}, and {\tt
  cvpi\_bmp\_header\_alloc\_write}, create a bitmap version 4 header structure
in heap memory and write it to the specified file.

All bitmap header elements are type {\tt uint32\_t}, {\tt int32\_t},
{\tt uint16\_t}, or {\tt int8\_t}, and it is 122 bytes long
\cite{BITMAPV4HEADER}. CVPI has support for multiple pixel encodings;
{\tt RGB\_565}, {\tt BGR\_565}, {\tt XRGB\_8888}, {\tt XBGR\_8888},
{\tt ARGB\_8888}, {\tt BGRA\_8888}, and {\tt RGBA\_8888}, where the
letters specify the channels and the numbers specify the number of
bits per channel. CVPI's {\tt cvpi\_avuy2argb} function requires {\tt
  ARGB\_8888}. {\tt CVPI\_BMP\_DEFAULT} macro specifies the correct
header type. Again, this has only been tested on a little endian
system and might not produce correct results on a big endian system.

\chapter{Video4Linux}
\label{sec-4} 
CVPI has a set of functions for starting, stopping, and getting data
from a camera. These functions can be found in {\tt
  cvpi\_camera\_setup.h} and {\tt cvpi\_camera\_setup.c} in the {\tt
  tests} directory. {\tt cvpi\_camera\_setup.c} is based off of the
file {\tt capture.c.xml}, which can be found in the Video4Linux system
documentation. {\tt cvpi\_camera\_setup.h} has four functions; {\tt
  cvpi\_camera\_create}, {\tt cvpi\_camera\_start}, {\tt
  cvpi\_camera\_read\_frame}, and {\tt cvpi\_camera\_takedown}.
Camera capture itself is outside the scope of CVPI, but it is necessary
for using CVPI in many computer vision applications.

{\tt cvpi\_camera\_create} requires the camera's width, height, and
format, as well as the number of image buffers to allocate in which to
dump raw image data. The function returns a structure on the heap that
gets passed to {\tt cvpi\_camera\_start}, which starts the camera,
turning on the camera light if it has one. To act on camera input, the
function {\tt cvpi\_camera\_read\_frame}, runs a given function on the
captured image. The function {\tt cvpi\_camera\_takedown}, stops the
camera and frees the camera information. Use the command line function
{\tt v4l2-ctl --all} to get information about what cameras are on the
system, and {\tt v4l2-ctl --list-formats-ext} to get information about
supported image formats and dimensions.

\chapter{Building}
\label{sec-8}
The project currently only builds using SCons. The current SCons build
is no more flexible than the original Makefile build; however, the
SCons build can be further improved and made more portable. SCons has
the full power of Python built in. SCons can also support
non-Unix-like systems such as Windows Mobile, something which GNU
Autotools cannot do easily \cite{whySCons}. Building with SCons is
nearly as easy as building with Make. Unfortunately, SCons has its own
shortcomings, not shared by Autotools or CMake
\cite{gentooSCons}. Autotools, CMake, and make cannot be used for the
same project; however, SCons can be used with those build systems
without interference, so it is likely that CVPI will continue to have
two build systems, one SCons and the other Autotools or CMake.

\section{Compiler Options}
CVPI has two levels of logging and three kinds of logging. CVPI must
be recompiled to switch between these. The C pre-processor variable
{\tt CVPI\_CAREFUL} if equal to 1 causes CVPI to check the return
values of all OpenVG commands with {\tt vgGetError}. Without it, only
commands that necessarily have to allocate GPU memory are checked,
though every OpenVG command could potentially cause a memory error
\cite[p.~29]{openvg}. If {\tt CVPI\_LOGGING} is set to 2, then logging
is done asynchronously; if it is set to 1, then logging is done
synchronously; else no logging is performed. CVPI could be built in
different directories with different logging settings and then the
client code linked depending on the user's needs. Little or no logging
makes the system harder to debug but reduces or removes the overhead
incurred by logging.

CVPI contains some inline assembly with alternative C code. To use the
assembly code, set {\tt CVPI\_ASSEMBLY} to 1.

There is also code specific to Broadcom's EGL API that is necessary
for EGL to run on the Raspberry~Pi. To enable the code, set
the pre-compiler variable {\tt HAVE\_BCM\_HOST} to 1.

\chapter{Using CVPI}
To use EGL, OpenVG, and CVPI, headers must be included in the correct
order. Simply including {\tt cvpi.h} should prevent such errors. The
EGL and OpenVG headers will be pulled in with it.

There are a few parameters common to many of OpenVG's image functions
that for CVPI to work correctly, must be set to certain values. The
parameters are {\tt outputLinear}, {\tt outputPremultiplied}, and {\tt
  allowedQuality}.  The parameter {\tt outputLinear} must always be
{\tt OUTPUT\_LINEAR}. The parameter {\tt outputPremultiplied} must
always be {\tt VG\_FALSE}. The parameter {\tt allowedQuality} must
always be {\tt VG\_IMAGE\_QUALITY\_NONANTIALIASED}. When creating an
image for use by a CVPI function, use {\tt CVPI\_COLOR\_SPACE}.

For creating and destroying a new EGL instance, CVPI provides a simple
method of doing so. In the following code in Figure~\ref{eglcreate}, {\tt
  cvpi\_egl\_surface\_pixmap\_native\_creator} and {\tt
  cvpi\_egl\_surface\_pixmap\_native\_destroyer} are names of user provided
functions, see Figure~\ref{surfacepixmap}. These functions are system
dependent. The code creates an EGL instance able to run CVPI on the
Raspberry~Pi.

\begin{figure}[H]
  \begin{lstlisting}[frame=single, basicstyle=\footnotesize]
cvpi_egl_settings settings = cvpi_egl_settings_create();

settings->surface_pixmap_create_function
          = cvpi_egl_surface_pixmap_native_creator;
settings->surface_pixmap_destroy_function
          = cvpi_egl_surface_pixmap_native_destroyer;
settings->renderable_api = cvpi_egl_renderable_api_openvg;
settings->current_surface_type = cvpi_egl_surface_type_pixmap;
/* additional settings may be required ... */
cvpi_egl_instance instance = cvpi_egl_instance_setup(settings);

/* OpenVG/CVPI code */

cvpi_egl_instance_takedown(instance);
free(settings);
\end{lstlisting}
  \caption{Creating and taking down an EGL instance using CVPI's interface.}
  \label{eglcreate}
\end{figure}

\begin{figure}[H]
\begin{lstlisting}[frame=single, basicstyle=\footnotesize]
/* the surface_pixmap_create_function function pointer type */
typedef EGLNativePixmapType (*pixmap_function_pointer)(cvpi_egl_instance);

/* function for surface_pixmap_create_function function pointer */
EGLNativePixmapType
cvpi_egl_surface_pixmap_native_creator(cvpi_egl_instance egl_instance)
{
  cvpi_egl_settings egl_settings_p = egl_instance->egl_settings;

  EGLint* pixmap_id = malloc(sizeof(*pixmap_id) * 5);
  if(pixmap_id == NULL) {
    return NULL;
  }

  pixmap_id[0] = 0;
  pixmap_id[1] = 0;
  pixmap_id[2] = egl_settings_p->width;
  pixmap_id[3] = egl_settings_p->height;
  pixmap_id[4] = egl_settings_p->pixel_format
                 | egl_settings_p->pixel_format_brcm;

  EGLint stride = cvpi_egl_bytes_per_pixel(egl_settings_p->pixel_format)
                                           * egl_settings_p->width;
  eglCreateGlobalImageBRCM(egl_settings_p->width, egl_settings_p->height,
			   pixmap_id[4], NULL, stride, pixmap_id);
  if(!(pixmap_id[0]) && !(pixmap_id[1])) {
    if(pixmap_id != NULL) {
      free(pixmap_id);
    }
    return NULL;
  }
  egl_instance->native_data = pixmap_id;
  return pixmap_id;
}

/* function for surface_pixmap_destroy_function function pointer */
EGLBoolean
cvpi_egl_surface_pixmap_native_destroyer(cvpi_egl_instance egl_instance)
{
  if(egl_instance->native_data != NULL) {
    EGLBoolean retval = EGL_TRUE;
    if(!eglDestroyGlobalImageBRCM(egl_instance->native_data)) {
      fprintf(stderr, "eglDestroyGlobalImageBRCM returned EGL_FALSE.\n");
      retval = EGL_FALSE;
    }
    free(egl_instance->native_data);
    egl_instance->native_data = NULL;
    return retval;
  } else {
    return EGL_FALSE;
  }
}
\end{lstlisting}
  \caption{Raspberry~Pi surface pixmap functions.}
  \label{surfacepixmap}
\end{figure}
\chapter{Sample Program}
The directory {\tt tests} contains a sample motion detection program
called {\tt cvpi\_sample\-\_motion\_detection}. There are two wrapper
shell scripts, {\tt cvpi\_sample\-\_motion\_detection\-\_data.sh} and
{\tt cvpi\_sample\-\_motion\_detection.sh}. The former program gathers
frames per second performance data at different resolutions. The
latter program is for user interaction, allowing the user to select
the resolution and number of frames to capture.

{\tt cvpi\_sample\-\_motion\_detection} uses Video4Linux to capture
image data from a web camera. The image data returned from the camera
is in YUYV format. CVPI converts the image from YUYV to YUVA. The
function {\tt cvpi\_image\_add} is used to subtract the current
captured image from the previously captured image, and also subtract
the previously captured image from the current captured image. The two
resulting images are OR'ed together using {\tt
  cvpi\_image\_logical\_or} with non-zero channel values set to 0 and
zero channel values set to 255. The resulting image is subtracted from
the captured image, so that values that did not change between frames
are zeroed and new values are kept. This image is then output to a
file. The file output is done asynchronously using a separate thread.

Data was gathered using {\tt
  cvpi\_sample\-\_motion\_detection\-\_data.sh}, but the data was somewhat
scattered. The program was run in an ssh terminal at run-level 3,
without the X.Org server running, so there was little else
running. Time samples were taken both inside and outside the program's
image capture and processing loop, with the lesser of the two samples
taken as the frame rate.  The average frame rate for all resolutions
tested was 17.9~fps with a standard deviation 5.9, and a maximum of
frame rate of 30.3~fps and a minimum of 12.1~fps. There was no real
correlation between the resolution and the frame rate. The result
could be due to the lack of a real time clock, the lack of a real time
operating system, or value overflow. Still, I think that the numbers
are not entirely inaccurate, and that you could expect process at around 16
to 18fps on average in simple vision applications.

\chapter{Bindings}
\label{sec-9}
CVPI is written entirely in C so it should not be too difficult to
create bindings in other languages. Currently there are bindings being
written for Chicken Scheme and C++ as part of CVPI. There are already
Java and Python bindings for OpenVG \cite{rPiVid}. Chicken Scheme's compiler
translates scheme to C code that then gets passed to a C
compiler \cite{chicken}. Making bindings in Chicken Scheme for C functions and
variables is quite easy. Scheme has the advantage of a more powerful
macro language and can be faster to write.

While C++ can use OpenVG and CVPI directly, but doing so does not allow
for use of C++ exception handling and memory management, so class
interfaces to OpenVG and CVPI were created.

The incentive for creating bindings as part of the project was due to
the size of the test code written in C and the amount of time spent
debugging it. Lisp was my natural choice. Lisp's macro system allows
code to be concise; common patterns that cannot be turned into
functions can be rolled into macros. Chicken Scheme has both Scheme
and Lisp-like macros systems. Traditional lisp macros can be written
with minor changes using {\tt ir-macro-transformer} and {\tt
  er-macro-transformer}, the former which implicitly renames variables
to prevent variable capture \cite[./man/4/Macros]{chicken}.

\chapter{Tests}
\label{sec-10}
Currently, there is a single testing program that gets compiled by
SCons in the {\tt tests} directory called {\tt cvpi\_tests}. The main
file, {\tt cvpi\_tests.c}, runs a list of test functions. A test
returns a CVPI\_BOOL value, and the test function name and its return
value are printed. Some of the tests check that the output is correct
but a number currently only check that the function runs.

I plan to move further testing to Chicken Scheme, and only use C for
testing that a function runs. Chicken Scheme's macro system should
make for smaller test code and it should be easier to do file IO in
Chicken than in C. The Chicken code will certainly introduce new bugs,
so it cannot completely replace testing in C.

\chapter{Coding Conventions}
\label{sec-11}
\section{Naming}
\label{sec-11-1}
All global CVPI function and variable names start with cvpi. This is
meant to serve the same role as a name-space. Following the name-space
is the the object type, and then the action. Actions are verbs
followed by adverbs. Object types are nouns followed by
adjectives. While this scheme does produce odd names from an English
speaking perspective, but it creates a logical classification system where
the type is primary and action is secondary. Words are separated by
underscores. Common CVPI object type names are {\tt channel}, {\tt
  color\_channels}, and {\tt image}. Channel functions only act on
one or two channels. {\tt color\_channel} functions act on the
color channels, and image functions act on all four channels.

Conversion functions are named such that a `2' separates the two
types.

OpenVG extensions follow OpenVG's naming convention of `vg'
followed by the rest of the name in camel case.

Constant macros are in all caps. General purpose function macros are
written in lower case. Boolean values and tests are in upper
case. Enumerations are in lower case. All structure and unions are
type-defined or have type-defined pointers to them.
\section{Code Structure}
\label{sec-11-2}
The overall CVPI project structure follows the GNU coding standards
required by GNU Autotools \cite{gnuSTD}.

All functions that allocate data on the heap or in GPU memory have a
function local jump point for freeing memory. Every function has its
own definition of the macro {\tt TAKEDOWN}, and every function has the
variable {\tt BADSTATE}. This is used to implement a kind of function
level deconstructor. At the beginning of the function, all heap
pointers are initialized to NULL, return values are initialized, and
all OpenVG ``objects'' are set to VG\_INVALID\_HANDLE. After every
OpenVG function, {\tt vgGetError} is called; and after every heap
memory allocation, the return value to checked for NULL. If memory is
freed outside of TAKEDOWN, the variable is set to NULL or
VG\_INVALID\_HANDLE. If an error value is returned, BADSTATE is set to
true and the function jumps into its TAKEDOWN code. If a memory
variable is non-NULL, it is freed, and if BADSTATE is true, then the
return value is set to the function's error return value, which is
VG\_INVALID\_HANDLE for functions that return VGImage's, NULL for
functions that return heap pointers, or else it is function
specific. The pre-compiler condition {\tt CVPI\_CAREFUL} allows the
user to turn off error checking of OpenVG functions that do not
necessarily involve memory allocation. According to the OpenVG
specification, every OpenVG function can throw a memory error;
however, memory errors are likely to only come from {\tt
  vgCreateImage} and {\tt vgDestroyImage}, and checking for errors
after every OpenVG call can needlessly slow down program execution.

All included headers are surrounded by {\tt ifndef} include guards. All
header files, if defined more than once, will print a compiler
message.

\section{Optimization}
\label{sec-11-3}
The code is written with the target platform being the ARMv6
CPU. Where possible, loops count down to zero. When comparing to
zero on the ARM CPU, a processor flag is set. Overall, this should
result in fewer instructions per cycle \cite[p.~180]{Langbridge}.

The GNU compiler has intrinsic C support for ARM's NEON SIMD
extensions; however, ARMv6 does not have NEON instructions but it
does have other parallel instructions that GCC does not have
intrinsic C support for. ARMv6 chips and higher have instructions
for parallel computations on two 16-bit and four 8-bit values in
normal registers. If these instructions are intrinsically supported,
they would be in the C header {\tt arm\_acle.h} \cite[p.~14]{IHI0053B}. GCC
does not have this header and the Clang compiler currently has an
incomplete version. Some inline assembly instructions were used
where GCC was unlikely to optimize correctly, specifically in the
{\tt cvpi\_avuy2argb} function. Single ARM instructions are capable
of multiple actions. Many instructions have a conditional flag that
will cause the instruction to be executed conditionally, and one of
the inputs passes through a barrel shifter before being passed to
the operator \cite[p.~37]{Langbridge}.

The function {\tt cvpi\_image\_coordinate\_table} returns a coordinate
table consisting of pairs of unioned {\tt uint16\_t}, and OpenVG writes
images to memory as a packed block with four 8-bit values per
pixel. This allows the user to take advantage of ARMv6's parallel
instructions. CVPI defines the union type, {\tt cvpi\_pixel}, with an
array accessor and channel index aliases {\tt cvpi\_pixel\_red},
{\tt cvpi\_pixel\_green}, {\tt cvpi\_pixel\_blue}, {\tt cvpi\_pixel\_v},
{\tt cvpi\_pixel\_u}, {\tt cvpi\_pixel\_y}, and {\tt cvpi\_pixel\_alpha}. This allows
the user to treat an image in memory as if it were a two
dimensional array, with the pixels being one dimension and the
channels being the other dimension.

\chapter{License}
\label{sec-12}
The code is licensed under the LGPLv3 (GNU Lesser General Public
License version 3); with the exception of {\tt convolve.el}, which is
licensed under the GPLv3 (GNU General Public License
version 3), and code in the {\tt m4} directory.

Graphics and data used in testing are licensed under the CC0 (Creative
Commons Zero) 1.0 Universal or any later version published by the
Creative Commons. All other graphics are licensed under the Creative
Commons Attribution 4.0 International License or any later version
published by the Creative Commons.

All documents are licensed under the GNU Free Documentation License,
Version 1.3 or any later version published by the Free Software
Foundation.

\bibliographystyle{plain}
\bibliography{sources}
\end{document}